\def\dMon{Mon, 01/18}
\def\dTues{Tues, 01/19}
\def\dWed{Wed, 01/20}
\def\dThur{Thur, 01/21}
\def\dFri{Fri, 01/22}
\def\dSat{Sat, 01/23}
\def\dSun{Sun, 01/24}
\placeDate


\begin{itemize}[noitemsep,topsep=0pt,leftmargin=*]
      \item \textbf{Lessons:} Software, Interface, Interaction, Modeling, Sculpting, Materials
      \item \textbf{Readings:} \due{Response Due \dTues}
            \begin{itemize}
                  \item \emph{\href{https://drive.google.com/file/d/1tnvKJBaXn2kFXjp-KB8p0_dkKuw1R4fu/view?usp=sharing}{Critical Response Process}}, Liz Lerman
                  \item \emph{\href{https://drive.google.com/file/d/1vUej0gK5nckVl6hvf68lcHJrh1w73h12/view?usp=sharing}{Pragmatics of Studio Critique}}, Judith Leeman {\footnotesize  (read pp. 181-190 up to ``Remainders'')}
                  \item \textbf{Prompt:} How do you define critique? What do you think are the most impactful ways of critique for students from diverse backgrounds? Can you see critique as dialogue or do you think they are two different strategies for engagement?
            \end{itemize}
      \item \textbf{Screenings/Artists:}
            \begin{itemize}
                  \item \href{https://alanwarburton.co.uk/goodbye-uncanny-valley}{\emph{Goodbye Uncanny Valley}}, Alan Warburton.
                  \item \href{https://www.nowness.com/story/virtual-embalming-frederik-heyman}{\emph{Virtual Embalming}}, Frederik Heyman \newline
                        \small{\ul{Trigger Warning:} Contains scenes depicting nudity and rope bondage.}
            \end{itemize}
      \item \textbf{Resources:} \href{https://cloud.blender.org/p/blender-fundamentals/}{Blender Fundamentals}, \href{https://sketchfab.com/search?features\=downloadable\&q\=scan+heritage\&sort\_by\=-relevance\&type\=models}{Sketchfab: 3D Foraging}
\end{itemize}