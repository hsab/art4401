\def\dMon{Mon, 03/08}
\def\dTues{Tues, 03/09}
\def\dWed{Wed, 03/10}
\def\dThur{Thur, 03/11}
\def\dFri{Fri, 03/12}
\def\dSat{Sat, 03/13}
\def\dSun{Sun, 03/14}
\placeDate

\begin{itemize}[noitemsep,topsep=0pt,leftmargin=*]
    \item \itemHead{Workshops:} Advanced Rigging, Advanced Materials, Audio Reactiveness, Advanced Drivers
    \item \itemHead{Proposal}[Project 2][Response Due \dTues]
          \begin{itemize}
              \item \textbf{A 9-panel storyboard}: Your storyboard can be either hand-drawn, digital, or prototyped using 3D software. If time permits, explore a combination of these methods to push your sketches closer to your mental image. Your storyboard should plan for \textbf{a minimum of 1 minute of animation}.
              \item \textbf{400-word written statement}: In your written response, explain your concept, creatures, and the environment that you have in mind. Explain the narrative arc of your animation project. How do you plan to execute future developments beyond what has already been accomplished? Which creatures will you use? How do they interact with one another?
              \item \textbf{References and mood board}: You should include a mood board (minimum 10 images/videos) in your proposal that that builds upon your existing work. Use these references to communicate your piece's mood, as well as materials and visual presentation.
              \item \textbf{Sounds}: Submit a minimum of 7 audio pieces that convey the mood and characterisitcs of your animation piece, creatures, and environment(s). Include these as links in your PDF file.
          \end{itemize}
          Submit your proposal as a PDF file under \discord{Proposals}{project-2}.
    \item \itemHead{Challenge}[Creature on fleek][Due \dSun]
          \newline Use the techniques covered during the week to improve the materials of your creature. Submit 3 renders from top, side, and front of your creature to our Discord server under \discordC
    \item \itemHead{Challenge}[Creature on the move][Due \dSun]
          \newline Use the rigging techniques covered during this week to create a walk cycle or other \emph{action} animations for your creature. Submit a short video of \emph{each} action to \discordC
    \item \itemHead{Screenings/Artists:}
          \begin{itemize}
              \item \emph{still lost I guess, here’s a tunnel\dots}, Darío Alva
              \item \emph{Fluid Silhouettes}, Jesse Kanda
              \item \emph{Trauma Scene 1}, Jesse Kanda
              \item \emph{Dream Playthrough \#3}, Sam Rolfes
              \item \emph{FLESH NEST}, Andrew Thomas Huang
              \item \emph{How Not to be Seen: A Fucking Didactic Educational .MOV File}, Hito Steyerl
          \end{itemize}
    \item \itemHead{Resources:}
          \begin{resenv}
              \begin{itemize}
                  \item \href{https://www.youtube.com/playlist?list=PLdcL5aF8ZcJv68SSdwxip33M7snakl6Dx}{\emph{Rig Anything with Rigify}, \emph{\href{https://www.youtube.com/playlist?list=PLdcL5aF8ZcJttvb-rgvyA1NkRi77zVP-J}{Advanced Rigify Techniques}}, Todor Nikolov }
              \end{itemize}
          \end{resenv}
\end{itemize}