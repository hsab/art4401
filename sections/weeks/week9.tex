\def\dMon{Mon, 03/08}
\def\dTues{Tues, 03/09}
\def\dWed{Wed, 03/10}
\def\dThur{Thur, 03/11}
\def\dFri{Fri, 03/12}
\def\dSat{Sat, 03/13}
\def\dSun{Sun, 03/14}
\placeDate

\begin{itemize}[noitemsep,topsep=0pt,leftmargin=*]
    \item \itemHead{Workshops:} Render Layers, Compositing, Video Editing, Motion Tracking
    \item \itemHead{Proposal}[Project 2][Response Due \dTues]
    \begin{itemize}
        \item \textbf{A 9-panel storyboard}: Your storyboard can be either hand-drawn, digital, or prototyped using 3D software. If time permits, explore a combination of these methods to push your sketches closer to your mental image. Your storyboard should plan for \textbf{a minimum of 1 minute of animation}.
        \item \textbf{400-word written statement}: In your written response, explain your concept, creatures, and the environment that you have in mind. Explain the narrative arc of your animation project. How do you plan to execute future developments beyond what has already been accomplished? Which creatures will you use? How do they interact with one another?
        \item \textbf{References and mood board}: You should include a mood board (minimum 10 images/videos) in your proposal that that builds upon your existing work. Use these references to communicate your piece's mood, as well as materials and visual presentation.
        \item \textbf{Sounds}: Submit a minimum of 7 audio pieces that convey the mood and characterisitcs of your animation piece, creatures, and environment(s). Include these as links in your PDF file.
    \end{itemize}
    Submit your proposal as a PDF file under \discord{Proposals}{project-2}.
    \item \itemHead{Exercise}[``Live Action'' Creature][Due \dSun] Place you creature in the motion tracked environment that we built during the workshops. Add additional elements to enhance, disrupt, or augment the existing environment from the video. Upload your video to Youtube or Vimeo and post the link to \discordE
    \item \itemHead{Challenge}[Cool Shadows][Due \dSun] Add realistic shadows to the exercise above. You can use Cycles a single shadow pass from Cycles, or use the complicated approach in EEVEE. If you do indeed complete this challenge, do not to resubmit the video, but simply post a still image that clearly shows your casted shadows to \discordC
\end{itemize}