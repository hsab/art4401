\def\dMon{Mon, 02/15}
\def\dTues{Tues, 02/16}
\def\dWed{Wed, 02/17}
\def\dThur{Thur, 02/18}
\def\dFri{Fri, 02/19}
\def\dSat{Sat, 02/20}
\def\dSun{Sun, 02/21}
\placeDate

\begin{itemize}[noitemsep,topsep=0pt,leftmargin=*]
        \item \textcolor{defaultColor}{\textbf{Open Studio Week}}
        \item \itemHead{Readings}[][Response Due \dTues]
        \begin{itemize}
            %   \item \emph{\href{https://drive.google.com/file/d/1tnvKJBaXn2kFXjp-KB8p0_dkKuw1R4fu/view?usp=sharing}{Critical Response Process}}, Liz Lerman
              \item \emph{\href{https://drive.google.com/file/d/1vUej0gK5nckVl6hvf68lcHJrh1w73h12/view?usp=sharing}{Pragmatics of Studio Critique}}, Judith Leeman {\footnotesize  (read pp. 181-190 up to ``Remainders'')}
              \item \textbf{Prompt:} How do you define critique? What do you think are the most impactful ways of critique based on the readings? Why would Leeman suggest to take the time to make ``obvious, verifiable observations'' about a piece of work? After reading Leeman's piece, how do you see the role of critique in your work or art practice? Leeman suggests how little trust the public has in their own experience of art viewing. ``A person fully capable of noticing and responding to a tree outside a gallery crosses the threshold into the gallery and becomes suddenly unable to muster that same capacity facing a work of art.'' Would you agree with this claim? Please explain your reasoning. Respond in 200-300 words. Submit your response to \discordR. Your responses and contributions to the discussion will be used during the class as we collectively create a guiding document on to approach critique as a class.
        \end{itemize}
\end{itemize}