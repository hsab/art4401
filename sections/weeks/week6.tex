\def\dMon{Mon, 02/15}
\def\dTues{Tues, 02/16}
\def\dWed{Wed, 02/17}
\def\dThur{Thur, 02/18}
\def\dFri{Fri, 02/19}
\def\dSat{Sat, 02/20}
\def\dSun{Sun, 02/21}
\placeDate

\begin{itemize}[noitemsep,topsep=0pt,leftmargin=*]
\item \textbf{Lessons:} Rigging, Audio Reactiveness, Drivers Refresher
\item \textbf{\textsc{Exercise 3} - Project 2 Proposal (Due \dFri)} Submit a proposal in PDF format for \hyperlink{project2}{Project 2}. As usual, this proposal is non-binding and subject to change. Nevertheless, it allows us to understand your concepts and ideas more tangibly. Your proposal should contain a minimum of:
          \begin{itemize}
              \item \textbf{A 9-panel storyboard}: Your storyboard can be either hand-drawn, digital, or prototyped using 3D software. If time permits, explore a combination of these methods to push your sketches closer to your mental image. Your storyboard should plan for \textbf{a minimum of 2 minutes of animation}.
              \item \textbf{400-word written statement}: In your written response, explain your concept, the emotional weight and the mood of the piece, how do you plan to execute it, and what software(s) are you planning to use. \newline
                    Example: ``I'm creating a piece about the excessive use of technology and its permeation in daily life\dots To achieve this my piece will primarily explore an imagined table filled with electronic, gadgets, and trash\dots I will be using Blender for my 3D needs and After Effects for post-processing and compositing\dots''
              \item \textbf{References and mood board}: You should include a mood board (minimum 10 images/videos) in your proposal that convey your vision more clearly. Use these references to communicate your piece's mood and environment, as well as materials and visual presentation.
              \item \textbf{Sounds}: Submit a minimum of 5 audio pieces that convey the mood of your animation piece. Include these as links in your PDF file.
          \end{itemize}
    \item \textbf{Readings:}
          \begin{itemize}
              \item \href{https://www.e-flux.com/journal/24/67860/in-free-fall-a-thought-experiment-on-vertical-perspective/}{\emph{In Free Fall: A Thought Experiment on Vertical Perspective}}, Hito Steyerl
              \item \textbf{Prompt:} How does the notion of perspective situate the observer in relation to reality? What other modes of observation could exist that value the perspective of the group above individuals? How does the vanishing point transform in such an optical apparatus? What other forms of meta-perspective can you imagine?\newline Think outside of the box, beyond images, think of volumes, X-Rays, or the residue of gunpowder as a form of observing a bullet in the past. Respond as a 400-word written essay, or better yet try to implement and visualize/render such ``perspectives''. (Caustics and volumetrics in Cycles can be your friend!)
              \item \textbf{Response Due \dThur}
          \end{itemize}
    \item \textbf{Screenings/Artists:}
          \begin{itemize}
              \item \emph{Fluid Silhouettes}, Jesse Kanda
              \item \emph{Trauma Scene 1}, Jesse Kanda
              \item \emph{Dream Playthrough \#3}, Sam Rolfes
              \item \emph{Ugly}, Nikita Diakur
          \end{itemize}
    \item \textbf{Resources:} \href{https://www.studiobinder.com/blog/how-to-make-storyboard/}{How to Make a Storyboard}
\end{itemize}