\def\dMon{Mon, 01/25}
\def\dTues{Tues, 01/26}
\def\dWed{Wed, 01/27}
\def\dThur{Thur, 01/28}
\def\dFri{Fri, 01/29}
\def\dSat{Sat, 01/30}
\def\dSun{Sun, 01/31}
\placeDate

\begin{itemize}[noitemsep,topsep=0pt,leftmargin=*]
    \item \textbf{Semester-long Project Proposals (Due \dTues)} This assignment only applies to \ul{those who are repeating this course for a second time}. If this is the first time you are taking 4401, this is NOT you! Moreover, if you are repeating this course and want to follow the standard syllabus this does not apply to you either. For repeating students who are pursuing a semester-long project, you must submit a proposal outlining your plan, concepts, ideas, and software. Your proposal must include:
          \begin{itemize}
              \item 500-word description of your idea and concept
              \item Include the approximate length of your animation.
              \item Rough timeline of deliverables and development plan for the next 13 weeks.
              \item A detailed storyboard, if your project is narrative work, or a mood board if your project does not follow conventional linear storytelling and narratives.
              \item A list of artists whose work you find inspiring in realizing your own project. These can be curated based on aesthetics, technique, software, and/or concept.
              \item List of the software you are using and how you plan to use them.
          \end{itemize}
          Submit your proposal as a PDF file under \textbf{Exercises $\Rightarrow$ \#extended-projects}. \newline
          \small{\textbf{Note:} As you progress through your project, we understand that things change. Creative work is part accident, part intention. This proposal enables us to better assist you in realizing your project and to follow and track your progress along the way. It is not a binding contract, so don't worry if things change.}
\end{itemize}
\vspace{1em}
\begin{itemize}[noitemsep,topsep=0pt,leftmargin=*]
    \item \textbf{Lessons:} Motion, Movement in Digital Software, Keyframes, Attributes, Camera Animation, Lighting, Basic Materials, Constraints, Drivers
    \item \textbf{\textsc{Exercise 2} - 3D Forage (Due \dTues)} Submit 3 renders of your kitbashed environment to be used for Project 1. In 4-5 sentences describe the motivation behind your choices and describe your concept.
    \item \textbf{Readings:}
          \begin{itemize}
              \item \href{https://reallifemag.com/motion-pictures/}{\emph{Motion Pictures}}, Patrick Nathan
              \item \textbf{Prompt:} Would you define Project 1 as an attempt in photography or rather a cinematic experience? Does it navigate space in the tradition of cinema, a freezing of time as do photographs, or does it echo a moment in time indefinitely? Why? What do you predict for the future of narrative in the age of 15-second videos, Instagram, and TikTok? Respond with a minimum of 200 words.
              \item \textbf{Response Due \dThur}
          \end{itemize}
    \item \textbf{Screenings/Artists:}
          \begin{itemize}
              \item \emph{Regular Division}, Joe Hamilton
              \item \emph{BREATHE DEEP}, Katie Torn
              \item \emph{insight}, Kim Laughton
          \end{itemize}
    \item \textbf{Resources:} \href{https://rhizome.org/art/artbase/}{Rhizome Artbase}
\end{itemize}