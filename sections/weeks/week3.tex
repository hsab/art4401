\def\dMon{Mon, 01/25}
\def\dTues{Tues, 01/26}
\def\dWed{Wed, 01/27}
\def\dThur{Thur, 01/28}
\def\dFri{Fri, 01/29}
\def\dSat{Sat, 01/30}
\def\dSun{Sun, 01/31}
\placeDate

\begin{itemize}[noitemsep,topsep=0pt,leftmargin=*]
    \item \itemHead{Workshops:} Materials, Environment Building, Kitbashing, Lighting, Rendering
    \item \itemHead{Prepare}[Alien Worlds][In-class \dTues] \newline Watch one or more episodes of \href{https://www.netflix.com/title/80221410}{Alien Worlds} and use this experience as a guiding force in conceptualizing and approaching your Project 1. Submit 3 rough sketches --- with your preferred medium --- for distinct creatures that you might want to pursue during the semester. Submit to \discord{Projects}{creature-sketches}
    \item \itemHead{Prepare}[Megascan Access][Due \dTues] Follow the guide on \discord{General}{resources} to prepare you account for \href{https://quixel.com/megascans}{Megascans} with unlimited access.
    \item \itemHead{Proposal}[Semester-long Project][Due \dThur] This assignment only applies to \ul{those who are repeating this course for a second time}. If this is the first time you are taking 4401, this is NOT you! Moreover, if you are repeating this course and want to follow the standard syllabus this does not apply to you either. For repeating students who are pursuing a semester-long project, you must submit a proposal outlining your plan, concepts, ideas, and software. Your proposal must include:
          \begin{itemize}
              \item 500-word description of your idea and concept
              \item Include the approximate length of your animation.
              \item Rough timeline of deliverables and development plan for the next 13 weeks.
              \item A detailed storyboard, if your project is narrative work, or a mood board if your project does not follow conventional linear storytelling and narratives.
              \item A list of artists whose work you find inspiring in realizing your own project. These can be curated based on aesthetics, technique, software, and/or concept.
              \item List of the software you are using and how you plan to use them.
          \end{itemize}
          Submit your proposal as a PDF file under \discord{Proposals}{extended-projects}. \newline
          \small{\textbf{Note:} As you progress through your project, we understand that things change. Creative work is ``part accident, part intention''. This proposal enables us to better assist you in realizing your project and to follow and track your progress along the way. It is not a binding contract, so don't worry if things change.}
    \item \itemHead{Exercise}[3D Forage][Due \dSun] Submit 3 renders of your environment and the creature placed within it based on your progress during the workshops and an additional 3 renders based on the changes that you have made to your model outside of class. If you are using resources outside of \href{https://quixel.com/megascans}{Megascans} credit them appropriately. Submit your 6 rendered images to \discordE
    \item \itemHead{Screenings/Artists}
          \begin{itemize}
              \item \emph{Regular Division}, Joe Hamilton
              \item \emph{BREATHE DEEP}, Katie Torn
              \item \emph{insight}, Kim Laughton
          \end{itemize}
    \item \itemHead{Resources:} \href{https://www.youtube.com/watch?v=whPWKecazgM}{\emph{World Building in Blender}, Ian Hubert}, \href{https://rhizome.org/art/artbase/}{Rhizome Artbase}
\end{itemize}