\section{Projects}

In light of the pandemic and its imposed limitations regarding space, fabrication, and occupying space, the projects have been designed to permit engagement with space without compromising your health, your peers', families', and faculties'. As such, projects that depend on utilizing physical space (projection mapping, augmented reality, immersive media) can be substituted with software-based versions that employ similar techniques in virtual spaces. To this end, those interested can use game engines such as Unity and Unreal Engine, as an architectural playground. We will discuss this together later in class and find compromises that fulfill the conceptual and theoretical requirements of the projects while allowing you to engage architecture, space, and animation in relation to these.

\subsection{Project 1: Inanimate, Animate}
Your first project is an environmental exploration from a first-person point of view. An inanimate landscape, with inanimate subjects. The only thing that moves is you: the creator, the observer, the camera.

In this project, you will begin by building a composite world within Blender or Cinema4D using the method of kitbashing. Kitbashing is a technique of world building that relies on various pre-existing \hyperlink{rsrc:models}{3D assets} that are composed on the fly in the 3D viewport. Kitbashing is the collage or montage for the 3D software. The world you build will rely heavily, if not all, on pre-existing \hyperlink{rsrc:models}{3D Models} that are improvised together. Think of how all the objects relate to each other in your world and how they contribute to a sense of place. Most importantly, be whimsical and fun with this part!

The main part of this project is to add and animate the lights and camera(s). Think of how your lighting, camera angles, location, and movement all contribute to telling a story about the world you've created.

To help guide you in this project, you can come up with a loose concept of place/space (i.e. a memory palace, ``a place where I would rather be'', a color-saturated grocery store, an apocalyptic train station, a floating castle, hell on earth, a hallucinogenic sand castle, a post-COVID reality, and so on). But allow your concepts to change as you kitbash and develop animations for your camera. The aim of this project is for you to think of \href{https://en.wikipedia.org/wiki/Mise-en-sc\%C3\%A8ne}{mise en scène} and cinematography within the 3D software.

If time permits, you can animate objects in your scene or you can also experiment with materials and textures. But your primary focus should be the mastery of camera animation and digital cinematography.

\textbf{Details:}
\begin{itemize}
	\tightlist
	\item Animation length of at least 1 minute
	\item Use of multipile cameras/shots
	\item Animated lighting and camera work
	\item Kitbashed and constructed environments
	\item Affective, provocative, and capable of evoking emotions/feelings in the viewer
\end{itemize}

\hypertarget{project2}{%
	\subsection{Project 2: Sonic Optics}}
Your second project is an exploration of sound and moving images in the context of 3D animation. It is also a preparatory step in realizing your final project, either by extension or serving as a technical and conceptual playground. For this project, you will exercise the production of 2 minutes of tightly audio-synchronized visuals. Unlike your first project, you are free from constraints in animating your scene(s). This means that you are in full control of your environments, 3D models, assets (regardless of their medium), cinematography, and lighting. Feel free to play with narrative or lack thereof. Much like your first project, Sonic Optics must be emotionally effective and transmissive, able to capture and evoke.

Beyond the conventional moving parts of realizing a work of animation and moving image, this project heavily relies on sound, audio, and music. Explore and experiment with the free sound resources that are posted on this website and our Discord server. Many of you might be quick to choose a piece of music. We encourage you to refrain from this approach and instead explore \href{https://en.wikipedia.org/wiki/Field_recording}{field recording} and layering found sound. You are certainly permitted to choose a piece of music, whatever it might be, however music should not be the primary component of your audio. A fantastic resource for exploring found audio and field recordings is \href{http://bbcsfx.acropolis.org.uk/}{BBC Sound Effects Library}. \href{https://splice.com/}{Splice} is another fantastic resource for finding pre-made audio effects. Beyond the available resources, we encourage you to experiment with field recordings of your own. Luckily, the barrier to entry for audio production and experimentation is much lower than computer graphics, and with a few hours of experimentation, you should be able to create enticing soundscapes.

\textbf{Details:}
\begin{itemize}
	\tightlist
	\item Animation length of at least 2 minute
	\item Use of multipile cameras/shots
	\item Animated lighting and camera work
	\item Kitbashed and constructed environments \textbf{OR} original assets
	\item Audio recorded/edited/generated by you with additional found sounds
	\item Video and audio closely complement/enhance eachother
\end{itemize}

\hypertarget{project3}{\subsection{Project 3: Open License}}

\textbf{Update:} Due to the unfortunate circumstances of the pandemic, the prospects of projection mapping, physical presentation, and equipment handling seem unlikely. As such we have to abandon the idea of projection mapping for this project.

Your final project is a condensation of the techniques, concepts, and theories that we have discussed in the class. Other than a few technical requirements, you are in full control of every aspect of your project. You are free to use any software/technique that you find appropriate for the realization of your work. Your final piece is a 3-5 minute animation which can be of any of the following forms: linear narrative, non-linear narrative, essay, experimental, or abstract.

You are responsible for the ideation, conceptualization, and execution of your work. Extra credit is given to those whose concepts relate to the readings or relevant contemporary issues. You must be able to discuss and contextualize your concept. In other words, you need to be able to appropriately discuss and justify your ideas. Moreover, the format of your work is also determined by you and can fall within or outside any of the following categories:
Non-photorealistic Animation (3D, 2.5D), 3D Animation, Mixed Media Collage, and Experimental Animation among others.

\textbf{Details \& Requirements:}
\begin{itemize}
	\tightlist
	\item \textbf{Length \& Format:}
	      \begin{itemize}
		      \item 2-3 minute
		      \item Minimum of 4 different shots/scenes
		      \item Export Settings: H.264 Codec | High Quality Profile | MP4 or MKV (Matroska) file format
		      \item Note: Render your projects as an image sequence for resumable renders and extra control. 
		      \item \href{https://en.wikipedia.org/wiki/Aspect_ratio_(image)#/media/File:Filmaspectratios.svg}{Aspect ratio} can be selected by you. Recommended: 2.35:1, 16:9, 4:3, 3:2, 1:1
	      \end{itemize}
	\item \textbf{Constructed in 3D Environment:} This point might seem confusing or vague. However, it simply implies that your work needs to be conceived and realized in a 3D space/software. Regardless of your chosen format, style, or form of narrative, you are required to construct your animation works such that you utilize depth in addition to the length and height of your image. Below are some examples of this idea in action:
	      \begin{itemize}
		      \item \textbf{2.5D Animation and Non-photorealistic Rendering}: \href{https://www.youtube.com/watch?v=Cl_EhU7elyo}{[1]}, \href{https://www.youtube.com/watch?v=tobqZ8fMqBk}{[2]}, \href{https://www.youtube.com/watch?v=C8puJClvNYE}{[3]},
		      \item \textbf{3D Animation}: \href{https://vimeo.com/297358261}{[1]}, \href{https://vimeo.com/385177134}{[2]}, \href{https://www.youtube.com/watch?v=8-ig_lnO7uU}{[3]},
		      \item \textbf{Mixed Media \& Collage}: \href{https://vimeo.com/167957360}{[1]}, \href{https://vimeo.com/hiradsab/outlier}{[2]}, \href{https://vimeo.com/203361631}{[3]}, \href{https://vimeo.com/181219125}{[4]}
		      \item \textbf{Experimental \& Abstract}: \href{https://vimeo.com/275668389}{[1]}, \href{https://vimeo.com/238456535}{[2]}, \href{https://vimeo.com/291430458}{[3]},
	      \end{itemize}
	\item \textbf{Sound \& Audio}: You are required to approach sound creatively and critically. Explore and experiment with the free sound resources that are posted on this website and our Discord server. We encourage you to explore and use your own \href{https://en.wikipedia.org/wiki/Field_recording}{field recordings}. Experiment with layering and designing your own audio. Creative and excellent sound work will be recognized and rewarded.
	\item \textbf{Quality and Effort:} Throughout the semester we have discussed a wide array of techniques, approaches, topics, and theories related to image-making and animation. Demonstrate your understanding and egagement with this material:
	      \begin{itemize}
		      \item Demonstrate animation techniques for lighting, materials, and objects
		      \item Demonstrate an understanding of keyframes, interpolation, and extrapolation and their appropriate use for different needs
		      \item Demonstrate an understanding of the used software and its affordances
		      \item Demonstrate your engagement with the readings and critical discussions
		      \item Demonstrate your understanding of the discussed theories and critical texts
	      \end{itemize}
	\item \textbf{Techniques and Approaches:} Your projects are required to demonstrate your abilities in \textbf{at least 6} of the following categories:
	      \begin{itemize}
		      \item \textbf{Worldbuilding and Kitbashing:} Limited to Megascans, 3D scans, and non-stylized models. Stylized models are not permitted.
		      \item \textbf{Cinematography:} Camera animation, scene construction, pacing, and editing.
		      \item \textbf{Rigging and Skeletal Animation:} Use of rigging and skeletal animation for characters and objects.
		      \item \textbf{Materials \& Shading:} Demonstrate abilities  in material creation, shading, and texturing.
		      \item \textbf{Non-moving Animation:} Demonstrate knowledge of animation for non-moving objects, such as lights, materials, and surfaces.
		      \item \textbf{Composting and Post-Production:} Use of post-processing and compositing for effects and colorgrading.
		      \item \textbf{Motion Tracking and VFX:} Mixing of live footage with 3D/2D. Use of photography and video to drive 3D animation. Hybrid video and 3D work.
		      \item \textbf{2.5D and NPR:} Use of 2.5D techniques to realize 2D, NPP, and anime-style animation.
		      \item \textbf{Audio Reactiveness:} Sound driven visuals, effects, and occurrences.
		      \item \textbf{Physics and Simulation:} Use of particle systems, fluid, smoke, rigid body, or soft body simulation among others.
		      \item \textbf{Procedural Animation \& Modifiers:} Use of modifiers and procedural means for object generation and animation.
	      \end{itemize}
\end{itemize}
