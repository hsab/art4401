\section{Projects}

In light of the pandemic and its imposed limitations regarding space, fabrication, and occupying space, the projects have been designed to permit engagement with the concepts and tools of this course without compromising your physical and mental health, your peers', families', and faculties'. To this end, those interested can use their preferred software, be it Cinema4D or game engines such as Unity and Unreal Engine, as well as other familiar software to explore and engage with this class. Although this course is primarily built around Blender, students are free to use their DCC of choice to fulfill the requirements of projects and assignments. However, do note that all of our activities, readings, assignments, and the course in general, are structured so that your efforts throughout the semester build towards your first and ultimately final project.

\subsection{Project 1: Meta Imaginaries}
The term \emph{xenomorph} originates from the Greek \emph{xeno-} meaning ``other'' or ``strange'' and \emph{-morph} which translates to ``shape''. Although the term has been used extensively in popular culture and science fiction, here, it functions in its literal meaning, "other-form". The idea behind this project came from Netflix's series \href{https://en.wikipedia.org/wiki/Alien_Worlds_(TV_series)}{Alien Worlds}: a heavily CGI docufiction that speculates on the possibilities of extraterrestrial "otherness." And similarly, for this project, you are tasked to imagine an otherworldly creature and its habitat.

Our work during the first five weeks of class culminates in the realization of this project. Your weekly exercises, challenges, readings, and discussions are designed to help you navigate and develop your projects. These activities not only satisfy their own requirements as individual assignments but also feed the progress towards completing Project 1. During this time, we will model/sculpt our xenomorphs, add materials and textures to them, build engaging environments and habitats for these creatures with control over lighting, and finally bring them to life by animating and rigging these characters. These activities not only push you forward in learning the techniques and standards of the software but also build up towards your Project 1. Week 6 is then dedicated to studio sessions where you can explore and enhance different aspects of your project before you submit them on Week 7.

\textbf{Details:}
\begin{itemize}
	\tightlist
	\item Animation length of at least 1 minute
	\item Use of multiple cameras/shots
	\item Animated lighting and camera work
	\item Animated character and creature building
	\item Shaded worlds and objects
	\item Kitbashed and constructed environments
	\item Affective, provocative, and capable of evoking emotions/feelings in the viewer
\end{itemize}

\hypertarget{project3}{\subsection{Project 2: Xenoorbis}}

Your final project is a continuation and condensation of the techniques, concepts, and theories discussed in the class. Other than a few technical requirements, you are in full control of every aspect of your project. You are free to use any software/technique that you find appropriate for realizing your work. Your final piece is a 2-3 minute animation that uses xenomorphs and environments built and shared by your peers to create a xenoorbis: other-worlds.

You are responsible for the ideation, conceptualization, and execution of your work. Extra credit is given to those whose concepts relate to the readings or relevant contemporary issues. You must be able to discuss and contextualize your idea. In other words, you need to be able to discuss and justify your ideas appropriately. Moreover, the format of your work is also determined by you and can fall within or outside any of the following categories: Non-photorealistic Animation (3D, 2.5D), 3D Animation, Mixed Media Collage, and Experimental Animation among others.

\textbf{Details \& Requirements:}
\begin{itemize}
	\tightlist
	\item \textbf{Length \& Format:}
	      \begin{itemize}
		      \item 2-3 minute
		      \item Minimum of 4 different shots/scenes
		      \item Export Settings: H.264 Codec | High Quality Profile | MP4 or MKV (Matroska) file format
		      \item Note: Render your projects as an image sequence for resumable renders and extra control.
		      \item \href{https://en.wikipedia.org/wiki/Aspect_ratio_(image)#/media/File:Filmaspectratios.svg}{Aspect ratio} can be selected by you. Recommended: 2.35:1, 16:9, 4:3, 3:2, 1:1
	      \end{itemize}
	\item \textbf{Constructed in 3D Environment:} This point might seem confusing or vague. However, it simply implies that your work needs to be conceived and realized in a 3D space/software. Regardless of your chosen format, style, or form of narrative, you are required to construct your animation works such that you utilize depth in addition to the length and height of your image. Below are some examples of this idea in action:
	      \begin{itemize}
		      \item \textbf{2.5D Animation and Non-photorealistic Rendering}: \href{https://www.youtube.com/watch?v=Cl_EhU7elyo}{[1]}, \href{https://www.youtube.com/watch?v=tobqZ8fMqBk}{[2]}, \href{https://www.youtube.com/watch?v=C8puJClvNYE}{[3]},
		      \item \textbf{3D Animation}: \href{https://vimeo.com/297358261}{[1]}, \href{https://vimeo.com/385177134}{[2]}, \href{https://www.youtube.com/watch?v=8-ig_lnO7uU}{[3]},
		      \item \textbf{Mixed Media \& Collage}: \href{https://vimeo.com/167957360}{[1]}, \href{https://vimeo.com/hiradsab/outlier}{[2]}, \href{https://vimeo.com/203361631}{[3]}, \href{https://vimeo.com/181219125}{[4]}
		      \item \textbf{Experimental \& Abstract}: \href{https://vimeo.com/275668389}{[1]}, \href{https://vimeo.com/238456535}{[2]}, \href{https://vimeo.com/291430458}{[3]},
	      \end{itemize}
	\item \textbf{Sound \& Audio}: You are required to approach sound creatively and critically. Explore and experiment with the free sound resources that are posted on this website and our Discord server. We encourage you to explore and use your own \href{https://en.wikipedia.org/wiki/Field_recording}{field recordings}. Experiment with layering and designing your own audio. Creative and excellent sound work will be recognized and rewarded.
	\item \textbf{Quality and Effort:} Throughout the semester, we have discussed a wide array of techniques, approaches, topics, and theories related to image-making and animation. Demonstrate your understanding and engagement with this material:
	      \begin{itemize}
		      \item Demonstrate animation techniques for lighting, materials, and objects
		      \item Demonstrate an understanding of keyframes, interpolation, and extrapolation and their appropriate use for different needs
		      \item Demonstrate an understanding of the used software and its affordances
		      \item Demonstrate your engagement with the readings and critical discussions
		      \item Demonstrate your understanding of the discussed theories and critical texts
	      \end{itemize}
	\item \textbf{Techniques and Approaches:} Your projects are required to demonstrate your abilities in \textbf{at least 6} of the following categories:
	      \begin{itemize}
		      \item \textbf{Worldbuilding and Kitbashing:} Megascans, 3D scans, stylized and non-stylized models, kitbashing.
		      \item \textbf{Cinematography:} Camera animation, scene construction, pacing, and editing.
		      \item \textbf{Rigging and Skeletal Animation:} Use of rigging and skeletal animation for characters and objects.
		      \item \textbf{Materials \& Shading:} Demonstrate abilities  in material creation, shading, and texturing.
		      \item \textbf{Non-moving Animation:} Demonstrate knowledge of animation for non-moving objects, such as lights, materials, and surfaces.
		      \item \textbf{Composting and Post-Production:} Use of post-processing and compositing for effects and colorgrading.
		      \item \textbf{Motion Tracking and VFX:} Mixing of live footage with 3D/2D. Use of photography and video to drive 3D animation. Hybrid video and 3D work.
		      \item \textbf{2.5D and NPR:} Use of 2.5D techniques to realize 2D, NPR, and anime-style animation.
		      \item \textbf{Sound \& Audio Reactiveness:} Sound driven visuals, effects, and occurrences.
		      \item \textbf{Physics and Simulation:} Use of particle systems, fluid, smoke, rigid body, or soft body simulation among others.
		      \item \textbf{Procedural Animation \& Modifiers:} Use of modifiers and procedural means for object generation and animation.
	      \end{itemize}
\end{itemize}
