\subsection{Catalog Description}

Studio art class exploring the relationship between sound and image in diverse fields of practice such as film, video art, music videos, live performances, and video games.

\subsection{Course Learning Objectives}

This course is an introduction and integration of sound and sonic compositions to advance multidisciplinary visual practices, design, communication, and art. Throughout the semester we will examine the language, context, and practice of sound production and how artists have contributed to and utilized this in their works. We will explore, compare, and contrast industry-standard/normative approaches with radical/experimental takes of various media in realtion to sound. Our aim is to establish a rich understanding of the complex and evolving environment in which artists and designers have been exploring the intimate realtions of sound and images. Students will explore technical, critical, and creative tools to realize audiovisual projects and to gain a deeper understanding of sound and media as a form of expression and communication.

In a series of weekly exercises, projects, presentations, discussions, workshops, and screenings students will explore and study the following:

\begin{itemize}
      \tightlist
      \item Principles and physics of sound: timing, composition, pitch, amplitude, timbre, etc
      \item Sound recording, editing, layering, and processing 
      \item Everyday sounds as compositional materials 
      \item Sound in multidisciplinary practices: visual art, cinema, performance, games, music, sculpture
      \item Dynamic relations between sound and image
      \item Effective use of audio in evoking and directing emotions 
      \item Sonic experiences through architecture, installation, and immersive media
\end{itemize}

\subsection{Health and Safety Requirements}

All students, faculty and staff are required to comply with and stay up to date on all \href{https://safeandhealthy.osu.edu}{university safety and health guidance}, which includes wearing a face mask in any indoor space and maintaining a safe physical distance at all times. Non-compliance will be warned first and disciplinary actions will be taken for repeated offenses.

\subsection{Format \& Delivery}

This is a hands-on, process-oriented studio. It is comprised of presentations, assignments, participatory activities and exercises, individual and group discussions, and reviews. This course is \hl{hybrid or in-person}. Synchronous Zoom meetings will be used for the introduction of assignments, some demonstrations, breakout group meetings, and group critique discussions. Other activities such as working on assignments and exercises, viewing videos, and reading assignments will be executed synchronously and asynchronously. In-person activities will include demonstrations, presentations, group exercises, and critiques. Weekly announcements will serve to inform when activities will take place.

\paragraph{Departmental Note:} A hybrid course provides online learning opportunities for up to 74\% of the semester. That means that up to three-fourths of your in-class meeting time may occur at a distance with the expectation that your full attention will be given to this course during the scheduled two hour and forty minute long meeting times, regardless if you are meeting physically or otherwise.

\subsection{Attendance}

Each unexcused absence (beyong the allowed three) will result in one full letter grade deduction (e.g. B+ to C+). Six unexcused absences (20\% of the semester) results in a failed grade. If there is an emergency and you must miss class, contact us beforehand. Absences will not be excused after the fact except in extreme circumstances. Illness requires a doctor’s note. If you are more than 10 minutes late, you will be marked tardy. Three tardies result in one unexcused absence. Any disputes should be discussed within two weeks.

\paragraph{Departmental Note:} The Department of Art acknowledges that illness, family obligations, and other conflicts with your classes do occur from time to time and up to three absences are allowed for any reason during the semester without penalty. \ul{All absences from class will be counted, however, and in the instance that you miss three class meetings, you are required to meet us to discuss strategies for avoiding additional absences}.

\paragraph{Departmental Note:} It has been determined that some in-person learning is necessary for you to successfully engage your instructor and peers, course activities, and to meet learning objectives. Timely and consistent contributions are critical in all formats used to deliver the content of this course. In the instance of class-wide quarantine or campus closure, a course contingency plan has been designed so that we can transition to an exclusively on-line format if we are required to actuate one. \ul{Attendance will be taken regardless of delivery format.}


\subsection{Participation}

Attendance, productive class activity and meeting in-progress deadlines are factors in the assessment of your progress. You are expected to be present and active for the entire class period. Participation is critical to passing and enjoying this class. Do the work, share your thoughts, ask questions, prepare for class meetings and discussions, offer feedback during critiques. This class is meant to be a safe space in which you feel encouraged and supported in learning and taking creative risks. This means being aware and considerate of different backgrounds, perspectives, and identities. Respect each other and this space we are building together. Don’t assume, ask. Remain open, be willing to take responsibility, apologize, and learn. Help each other in this. If you have concerns, please let us know.

% \subsection{Communication}\label{ssec:communication}
\hypertarget{communication}{%
      \subsection{Communication}}

\href{http://discordapp.com/}{Discord} is used as our primary mode of communication. You are required to signup for an account, join our \href{\discordURL}{server}, and keep up to date with announcements and group discussions. Discord is also used to organize resources, readings, screenings, and learning materials. Here, you will also submit your assignments.

\subsection{Discord Server Interaction}
Ongoing weekly discussions and participation in the Discord \href{\discordURL}{server} is required. We will use Discord to gather and share resources, respond to readings and peers' works, and to share your work in progress.

Each week should feature at least:
\begin{itemize}
      \tightlist
      \item Link to your exercise/project with a short description of your learning process, concept, challenges, and triumphs. This way your work is contextualized for your peers in relation to your creative inputs and the readings.
      \item Respond to at least two of your peers' exercises and project submissions.
\end{itemize}

\subsection{Readings \& Discussions}

During the quarter, you will be assigned readings on a variety of topics. The readings are intended to familiarize you with some of the relevant discussions that relate to the field. We will discuss our findings and thoughts with our peers in class. Your participation in these discussions matters. The discussions serve as a dialectical engagement to learn from one another and explore the readings in conversation. Moreover, the readings serve as a foundation for discussing the screenings, which are purposefully picked to convey some of the ideas from the readings in practice.

\subsection{Projects}

Projects are due at the start of class on the date assigned. Projects may be turned in up to one week late for a one letter grade deduction off the project grade. Work that is more than one week late will not be accepted. If you are absent, you are still expected to turn in projects online by the deadline. Extra time will not be given for work lost due to save issues, software errors, computer crash, etc. You should regularly backup your files on your desktop, online, and/or on an external harddrive or USB stick in case your computer is lost.

\subsection{Grading}

There are 100 possible points, distributed across participation, attendance, exercises, and projects. Individual works will be assessed according to assignment objectives, effort and quality of in-class and online or distance activities, vigor of exploration and research initiative, participation in reviews and discussions, and ability to adapt.

\hspace*{1em} Participation \& Discord Interaction: 15 pts\\
\hspace*{1em} Ecercises: 15 pts\\
\hspace*{1em} Project 1: 20 pts\\
\hspace*{1em} Project 2: 20 pts\\
\hspace*{1em} \ul{Project 3: 30 pts}\\
\hspace*{1em} Total: 100 pts

\subsection{Late Assignments}

If you miss deadlines due to valid, extenuating circumstances you may submit the required work at a date agreed upon with us. Please contact us to discuss modifying the deadline prior to the original deadline.

\subsection{Grading Scale}

\begin{tabularx}{\textwidth}{@{}l @{}l X@{}}
      A \hspace*{1em} & (93 - 100) & Work, initiative, and participation of exceptional quality             \\
      A-              & (90 - 92)  & Work, initiative and participation of very high quality                \\
      B+              & (87 - 89)  & Work, initiative and participation of high quality                     \\
      B               & (83 - 86)  & Very good work, initiative and participation                           \\
      B-              & (80 - 82)  & Slightly above average work, initiative and participation              \\
      C+              & (77 - 79)  & Average work, initiative and participation                             \\
      C               & (73 - 76)  & Adequate work; less than average level of initiative and participation \\
      C-              & (70 - 72)  & Passing but below good academic standing; less than average level      \\
      D+              & (67 - 69)  & Below average work, initiative and participation                       \\
      D               & (60 - 66)  & Well below average work, initiative and participation                  \\
      E               & (59.9 - 0) & Unsuccessful completion of work. Limited or no participation.
\end{tabularx}


\subsection{Course Technology}

\begin{itemize}
      \tightlist
      \item Basic computer and \href{https://lmgtfy.com/}{web-browsing skills}
      \item Navigating Carmen: for questions about specific functionality, see the \href{https://community.canvaslms.com/docs/DOC-10701}{Canvas Student Guide}.
      \item \href{https://go.osu.edu/Bqdx}{CarmenZoom Virtrual Meetings}
\end{itemize}

\subsection{Required Equipment}

\begin{itemize}
      \tightlist
      \item Computer: OS X, Windows 7+, or Linux with internet connection for CarmenZoom\\
            Recommended Hardware:
            \begin{itemize}
                  \item 64-bit quad core CPU
                  \item 16 GB RAM
                  \item Full HD display
                  \item Drawing tablet
                  \item Graphics card with 4 GB RAM
            \end{itemize}
      \item 3-button mouse
      \item Webcam
      \item Microphone
      \item A mobile device (smartphone or tablet) or landline to use for BuckeyePass authentication
\end{itemize}

\subsection{Course Materials and Tools}

Our course heavily relies on free, open-source, and libre software. Throughout the semester we will explore modeling, rendering, and animation using \href{http://blender.org/}{Blender}, \href{https://www.adobe.com/products/aftereffects.html}{Adobe After Effects}, \href{https://www.unrealengine.com/en-US/}{Unreal Engine}, \href{https://unity.com/}{Unity}, and \href{https://www.daz3d.com/}{DAZ}, among others. Blender provides a powerful arsenal of tools that enables advanced 2D and 3D exploration, animation, video editing, and compositing among others. Students that are already familiar with other 3D suites such as Cinema4D and Maya are ecnouraged to use their software of choice. Unity and Unreal Engine are game engines using which we will disseminate interactive works of 3D animation.

You are required to signup for an account on \href{http://sketchfab.com}{Sketchfab}, an online 3D model sharing platform. Here you will post your animated 3D models and scenes for assessment and dissemination among your peers. Sketchfab is also used as an AR platform for exercises and projects.

\href{http://discordapp.com/}{Discord} is used as our primary mode of communication. You are required to signup for an account, join our \href{\discordURL}{server}, and keep up to date with announcements and group discussions. Discord is also used to organize resources, readings, screenings, and learning materials. Here, you will also submit your assignments.

You are required to signup for Youtube. The platform is used to share your 3D animations in video format. You will create an unlisted playlist, to which your peers and us have access to view weekly exercises and larger projects.

All required readings and screenings will be posted on our Discord \href{\discordURL}{server}. There is no required book for this class. We will coordinate and discuss with the department the possibilitites of lab computer use. However, given our current post-COVID reality, this course is structured such projects and exercises can be completed with consumer-grade PCs and laptops.

\textbf{This course requires a 3-button mouse (left, right, clickable wheel) and a computer \href{https://www.blender.org/download/requirements/}{capable} of running Blender (or your software of choice).}